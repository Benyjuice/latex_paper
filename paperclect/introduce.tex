\section{引言}
\subsection{水声通信发展历程}
当今世界已进入了飞速发展的信息时代,通信
是这一进程中发展最为迅速、进步最快的行业。陆地
和空中通信领域包括的两个最积极、最活跃和发展
最快的分支———互联网和移动通信网日臻完善,
而海中通信的发展刚刚崭露头角。有缆方式的信息
传输由于目标活动范围受限制、通信缆道的安装和
维护费用高昂以及对其他海洋活动( 如正常航运)
可能存在影响等缺点,极大地限制了它在海洋环境
中的应用。另外由于在浑浊、含盐的海水中,光波、电
磁波的传播衰减都非常大,即使是衰减最小的蓝绿
光的衰减也达到了40dB/km,因而它们在海水中的
传播距离十分有限,远不能满足人类海洋活动的需
要。在非常低的频率(200Hz以下),声波在海洋中却
能传播几百公里,即使20kHz的声波在水中的衰减也只有2-3dB/km,因此水下通信一般都使用声波
来进行通信。而在这个频率范围内,声波在水中( 包
括海水)的衰减与频率的平方成正比,声波的这个
特性导致了水下声信道是带宽受限的。采用声波作
为信息传送的载体是目前海中实现中、远距离无线
通信的唯一手段\upcite{woodward1996digital}。

水声通信最初是主要应用于军事领域,最近十
年,随着人类对海洋资源的不断开发和利用,如近年
来环境系统的污染监测、海上石油工业的遥控以及
不回收仪器设备而直接获取海底工作站记录的科学
数据等等,使得水下信息通信技术的商用前景越来
越广阔,相应地,也促进了水下通信技术的发展。早
在第二次世界大战之后,水声通信就已出现。现在水
声通信已广泛应用于潜艇之间的通信、水面舰艇与
潜艇的通信、海上遥控和遥测、遥感、水下测量设备
记录数据回送、水下图像传输、语音传输和水声局域
互联网(ALAN)等多个方面。其具体的设备包括舰
艇用水声通信机、水下通信浮标、水下应急通信设
备、潜水员水声电话、水下机器人用的图像声纳和通
信声纳等许多种\upcite{goodfellow1977underwater}。

水声通信技术诞生于上世纪中叶,和其他信号
处理技术的发展趋势相同,也经历了从最初的模拟
通信阶段到现如今的数字通信阶段的过程。总的来
说,水声通信,特别是高速水声通信,近十几年的发
展趋势是由非相干通信向相干通信发展,并且随着
硬件水平、信号处理芯片计算能力的不断提高,水声
通信的调制方式、信号处理算法等都在逐渐使用各
种新的、复杂的技术,比如空间调制技术、自适应均
衡技术、盲均衡技术、分集接收技术等\upcite{catipovic1989acoustic,stojanovic1996recent,proakis1991adaptive,hinton1992adaptive}。

现在,水声通信已经发展到建立水声网络的阶
段\upcite{johnson1994design,lapierre2001design,yeo1999analysis,xie2001network}。
当前水声通信的目标是建立水下自治采样
网络(AOSN)。这种网络能够提供多个网络节点间
交换数据的功能,与此同时,也己经提出了能够传输
包括图像、数据、控制命令、语言等多种信息的水声
局域网络协议\upcite{蔡惠智2006第八讲}。

 






\subsection{水声水声扩频通信发展历程}

\subsection{国内外水声扩频研究现状}